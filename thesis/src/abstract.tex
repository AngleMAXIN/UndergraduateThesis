\newpage

~~\par
~~\par

\renewcommand\abstractname{\heiti\zihao{3}摘~~要}

\begin{abstract}\zihao{-4}
Online Judge(在线评测系统,以下简称OJ)最初是用作ACM选手的训练平台。用户登录系统提交相关题目的源代码后,系统就会进行编译、运行和评测,然后返回结果。计算机及相关专业的高级编程语言课程、算法设计课程、数据结构课程等也可以用OJ来作为练习和考试的平台。

从OJ系统的需求分析入手,讨论了从系统架构设计到具体功能开发以及部分遇到的问题和挑战,逐步的实现一个完整功能的OJ,包括基于Python和Django的Web程序开发,基于avalon的复杂单页面网页的构建,数据库和缓存的性能调优,基于seccomp的沙盒安全机制的设计,基于Docker的部署运维等。

\heiti{关键词}~~在线评测系统~~ACM竞赛~~沙箱~~Python~~单页面程序
\end{abstract}

~~\par
~~

\renewcommand\abstractname{\zihao{3}Abstract}

\begin{abstract}\zihao{-4}
Online Judge (OJ) is intended to train ACM players originally. After logging into the system, user can submit solutions of the problems, OJ system will compile and run the code and then return the result. Meanwhile, for courses like Programming Languages,Algorithm and Data Structures, OJ is a great platform to learn and practice skills.

The paper will start with the requirement analysis of OJ system, and then we will discuss topics ranging from system architecture design, specific module development to issues and challenges we solved, including Web server development based on Python and Django, database and cache related performance optimization, complex single page application based on avalon, sandbox security mechanism based on seccomp.

\textbf{Keywords}~~OnlineJudge~ACMContest~Sandbox~~Python~SinglePageApplication
\end{abstract}